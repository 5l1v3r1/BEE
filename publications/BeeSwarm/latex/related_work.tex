\section{Related Work}
\label{related_work}
Scalability is one of the most important metric we evaluate the quality of HPC applications. Many works have been done to build scalability test tools to facilitate HPC application development. For example, \cite{vetter2005mpip} proposed a a lightweight profiling library for MPI applications, which only based on statistical information about MPI functions and brings little performance overhead. \cite{chen2006stas} proposed a effective scalability testing and analysis system -- STAS. \cite{chung2006mpi} proposed a configurable MPI scalability analysis tool for Blue Gene/L supercomputer. \cite{brunst2013custom} proposed a performance too, Vampir, that can be used to detect hot spots in HPC applications. This can efficiently help HPC developers make their applications more scalable. \cite{merchant2012tool} proposed JACE (Job Auto-creator and Executor), a tool that enables automation of creation and execution of complex performance and scalability regression tests. It can help developers tune an application on a given platform to maximize performance given different optimization flags and turnable variables. \cite{muraleedharan2012hawk} presented a HPC performance and scalability test tool, Hawk-i, that uses cloud computing platforms to test HPC applications in order to reduce the effort to access relative scarce and on-demand high performance resources. \cite{bell2003paraprof} proposed, ParaProf, a portable, extensible, and scalable tool for parallel performance profile analysis. It gathers rich number of hardware counters and traceable information in order to offer much more detailed profiling result similar to state-of-the-art single process profiling tools. \cite{yoo2015patha} proposed a scalability test tool, PATHA, that uses system logs to extract key performance measures and apply the statistical tools and data mining methods on the performance data to identify bottlenecks or to debug the performance issues in HPC applications. Although these recent works are useful in scalability test for HPC applications, their tools or systems cannot be easily adopted by current HPC application development projects since they either require modification to the HPC application or complicated installation or configuration process in order to make their tools working properly on a given HPC platform. 
