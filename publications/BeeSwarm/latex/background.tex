

\section{Background}
\label{background}
\subsection{Build and Execution Environments (BEE)}
\texttt{BEE} \cite{bee,beeflow} is a %\trandles{remove 'Docker-based' because BEE can use multiple container runtimes} 
containerization environment that enables HPC applications to run on both HPC and cloud computing platforms. \texttt{BEE} provides a unified user interface for automatic job launching and monitoring. \texttt{BEE} users only need to wrap their applications in a standard Docker image and provide a simple \texttt{BeeFile} (job execution environment description) to run on \texttt{BEE}. Since the same Docker image is used %\trandles{change to "Docker image is used"} 
 across platforms, no source code modification is necessary. %In addition, \texttt{BEE} solves the security constraint of HPC environment that cannot be addressed with current Docker daemon.%\pat{isn't this off I'm not sure most people will think of Docker as secure}
 %\trandles{I agree with Pat. BEE doesn't solve using Docker securely in an HPC environment}
In this work, we build \texttt{BeeSwarm} based on \texttt{BEE}, so it naturally inherits all benefits of \texttt{BEE}. This allows us to build a unified scalability test system across multiple platforms. 



\subsection{Continuous Integration (CI)}
CI was first named and proposed by Grady Booch in 1991. Its aim was to greatly reduce integration problems. CI was initially combined with automated unit testing to run on the developer's local machine before committing to the central code repository.  However, as software being developed becomes more complicated and more people are involved in developing, localized testing becomes inefficient and the code base on each developer's machine can easily become outdated, so integration can still be problematic. The longer a branch of code remains checked out, the greater the risk of multiple integration conflicts and failures when the developer branch is reintegrated into the main line. So, centralized build servers are used for CI. The build servers can perform more frequent (e.g., every commit) test runs and provide reports back to the developers. Driven by these benefits many HPC application development projects are now using CI. For example, almost all projects in Next-Generation Code Project in Los Alamos National Laboratory are using CI \cite{daniel2016lanl}. Currently, many CI tools are available to developers such as Travis CI, Circle CI, Codeship, etc. Many computing platforms also provide CI as a feature in their services such as AWS, Azure, etc. However, current designs of CI services only focus on detecting software bugs in the HPC softwares. To the best of our knowledge, none of the current work can easily enable automatic scalability tests in CI. So, in this work we propose to enable easy scalability tests for HPC developers. %\paul{Is it valuable to reference studies that have sought to analyze the additional value provided via CI or has it been generally accepted that CI is a good thing?} %\pat{it couldn't hurt}
