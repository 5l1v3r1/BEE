\section{Backgrounds}
\subsection{BEE}
\texttt{BEE} \cite{bee} is a Docker-based containerization environment that enables HPC applications to run on both HPC and cloud computing platforms. \texttt{BEE} provides a unified user interface for automatic job launching and monitoring. \texttt{BEE} users only need to wrap their application in a standard Docker image and provide a simple \texttt{BeeFile} (job execution environment description) to run on \texttt{BEE}. Since the same standard Docker environment is provided across platforms, no user application modification is necessary. In addition, \texttt{BEE} solves the security constrain of HPC environment that cannot be addressed with current Docker daemon. In this work, we build \texttt{BeeFlow} based on \texttt{BEE}, so it naturally inherits all benefits of \texttt{BEE}. This allows us to build a unified workflow management system across multiple platforms. 
\subsection{In situ Analysis}
In traditional scientific workflows, data are usually shared via file systems between tasks. However, as we are aiming to solve more complicated problems, workflow data can reach hundreds of terabytes to petabytes. It can be inefficient to store large amounts of data on disks for extended periods. One solution to mitigate this problem is in situ analysis \cite{sewell2015large}, allowing data consuming tasks to run along with the data producing tasks. As data producing tasks make progress, they can send the intermediate data to data consuming tasks for real-time processing and output results simultaneously. The data can be transfered either via smaller temporary or permanent data files on shared filesystems or via network communication between processes. This type of workflow enables users to identify potential problems during current simulations and further reconfigure, adjust, and restart simulations to save time; get flexible real-time simulation control; or filter out less relevant data as needed to save computing resources. In order to launch in situ workflows, current users need to manually launch each task, which introduces much complications. However, none of the current workflow management systems natively support in situ workflows, therefore in this work, we propose to design a workflow launcher with in situ analysis support.