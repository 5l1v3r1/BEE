\section{Related Work}

In this section, we discuss and compare several commonly used scientific workflow management systems. Those have been intensively used by various fields, including biology, computational engineering, astronomy, and climate modeling. Most of them share similar five-layer functionality architecture, so that they have similar features. However, each of them still has some different designs and functionalities as discussed in each of the following subsections.

\subsection{Pegasus}
Pegasus \cite{deelman2005pegasus} is a commonly used scientific workflow management system that has been adopted in many scientific research fields. Similar to \texttt{BeeFlow}, Pegasus features portability and stability on different infrastructures including HPC and cloud platforms. It also features, data transfer support across different execution partitions and fault-tolerance. On the other hand, since \texttt{BeeFlow} aims to support in situ analysis, either involving multiple application-specific file generation or network/in-memory data transfer, we choose to let users handle the file transfers between tasks or simply use our shared storage or SSH tunnel options. As for fault tolerance, many in situ analysis enabled scientific applications feature in-application fault tolerance or checkpoint/restoration, so we leave handling fault tolerance to the application. Moreover, in the presentation layer Pegasus uses XML-format DAG for user workflow input, on the other hand, \texttt{BeeFlow} uses JSON-formatted configuration files, providing similar usability. Finally, in the user service layer, both Pegasus and \texttt{BeeFlow} provide workflow status monitoring. However, Pegasus does not support Docker images as user input, so potential software compatibility issues need to be handled by Pegasus users.

%\subsection{Swift}
%Swift \cite{zhao2007swift} is another scientific workflow management system similar to Pegasus. It also features portability across HPC and cloud platforms. In its presentation layer, Swift usess its own language, SwiftScript, to let users define their workflow logics. As for user service layer, Swift only provides provenance data monitoring without execution monitoring. After WEP generation, the execution is done by Karajan workflow execution engine, that is responsible for task instantiation, scheduling, data transfer and communicate with underlying computing infrastructure.

\subsection{Kepler}
Kepler is a scientific workflow management system based on the Kepler project \cite{altintas2004kepler, altintas2004kepler2}. Like \texttt{BeeFlow}, Kepler can also utilize both HPC and cloud computing resources. Kepler also allows users to plug in different execution models into the workflow. This feature can potentially support in situ analysis dependencies. However, it requires intensive development effort, and special dataflow in in situ workflow can be hard to adopt. For the presentation layer, Kepler integrates a graphical workbench as the user interface. Each task in Kepler is called an actor, a highly reusable module for workflow design. After designing the workflow, Kelper uses another component named director to generate an executable workflow then launches the workflow with either static or dynamic scheduling \cite{bux2013parallelization, ludascher2006scientific}. Kepler also implemented three kinds of system level fault tolerance mechanisms.

\subsection{Taverna}
Taverna \cite{missier2010taverna} is a scientific workflow management system specifically designed for biological experiments. Like Kepler, Taverna has a graphical user interface in the presentation layer to help users design and input their workflow. This user interface also serves as a monitor for workflow status. Taverna can be installed on either users' local machines or web servers and can utilize both HPC and cloud computing resources. However, Taverna does not have the flexibility to integrate different execution models to support in situ workflows.

\subsection{Chiron}
Chiron \cite{ozsu2011principles} a data-intensive scientific workflow that exploits a database approach. In its presentation layer, it allows users to define workflows with an algebraic or SQL expression. Then each expression will be converted into six basic operations: Map, SplitMap, Reduce, Filter, SRQuery, and MRQuery. In the user services layer, workflow monitoring and steering features are provided. Like \texttt{BeeFlow}, Chiron can also utilize HPC and cloud platforms. The cloud computing part is done by Scicumulus \cite{de2010scicumulus, oliveira2012adaptive}. Similar to other workflow management systems we discussed, Chiron does not support Docker containers, thus users need to handle compatibility issues. in situ workflows are also not supported in Chiron.

%\subsection{Galaxy}
%\subsection{Triana}
%Triana \cite{taylor2007triana} is a scientific workflow management system similar to Kepler. It also features a graphical user interface in presentation layer. Each components in its workflow is similar to the actor in Kepler. For workflow execution, Triana can use both HPC cluster and cloud computing resources. On the cloud, it utilizes RabbitQM12 for optimized communication between VMs.
%\subsection{Askalon}

